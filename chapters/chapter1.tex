%\tableofcontents{}
\chapter{Introduction}
\label {sec:introduction}
Computer vision, a relative new area of research has risen with the growth of technology for the past decade. It is an emerging science which involves teaching machines or computers to see and make decisions and judgements. As discussed in \cite{EGL}, goals of computer vision can be divided into different categories as engineering and way to understand intelligence. Real world applications like surveillance, photography and navigation have been developed and put into use gradually. By applying machine learning techniques, applications can be designed and built to meet varies needs. Although computer vision has a lot of potential applications and is much more efficient than human as the digital data size grows tremendously every year, it has been considered difficult because it frequently fails in accuracy to human visual system by comparison. 

For instance, a human can distinguish between body parts of different people under almost any circumstances which are not too extreme. However, a computer may not be able to easily achieve this kind of task due various conditions such as light illumination, viewpoints and different gestures. Therefore, how to represent human knowledge in computers and carry out real time computation efficiently and accurately have become the biggest challenges in the area of computer vision. Moreover, vision techniques are very domain dependent. Some approaches may fail in one specialized application but work well in other specified areas. \cite{EGL}. 

Face is a key subject in the area of object detection. Especially face recognition, which is an technology used to verify people's identity is very widely used in various fields such as security authentication systems, search potential criminals in urban area and unlocking smart phones. And research conducted on this subject is getting more and more important. In this project, we will try to address some sub-problems in facial recognition.

This report will emphasis on the state of the art machine learning algorithm known as random forest, which will be mentioned in detail in section \ref{sec:RF} and will be used to tackle problems such as head pose estimation in section \ref{sec:HPestimation} and face feature points detection in section \ref{sec:FLL} respectively.